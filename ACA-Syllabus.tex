\documentclass[letterpaper,oneside,10pt]{scrartcl}

\usepackage[margin=1in]{geometry}
\usepackage[T1]{fontenc}
\usepackage[ansinew]{inputenc}
\usepackage{lmodern} %Type1-font for non-english texts and characters
\usepackage{microtype} %Type1-font for non-english texts and characters
\usepackage{graphicx} %%For loading graphic files
\usepackage{amsmath}
\usepackage{amsthm}
\usepackage{amsfonts}
\usepackage{url}
\usepackage{xcolor}
\usepackage{hyperref}

\renewcommand{\familydefault}{\sfdefault}
\setlength{\parindent}{0mm}
 



%%%%%%%%%%%%%%%%%%%%%%%%%%%%%%%%%%%%%%%%%%%%%%%%%%%%%%%%%%%%%
%% DOCUMENT
%%%%%%%%%%%%%%%%%%%%%%%%%%%%%%%%%%%%%%%%%%%%%%%%%%%%%%%%%%%%%
\begin{document}

\title{MUSI 6201/4457: Audio Content Analysis}
\author{syllabus }
\date{Fall 2023} %%If commented, the current date is used.
\maketitle

\pagestyle{plain} %Now display headings: headings / fancy / ...

\section*{Course Details}
    \begin{tabular}{ll}
        \textbf{class time} & MW 3:00--4:15pm \\
        \textbf{location} & West Village 163\\
        \textbf{credits} & 3 credit hours
    \end{tabular}
\section*{Instructor Information}
    \begin{tabular}{lp{70mm}l}
        \textbf{} & \textbf{instructor} & \\%\textbf{teaching assistant}\\
        \textbf{name} & Alexander Lerch & \\%Ashis Pati, Siddharth Gururani \\
        \textbf{email} & \url{alexander.lerch@gatech.edu} & \\%\url{{ashis.pati,siddgururani}@gatech.edu} \\
        \textbf{location} & Couch 205 (840 McMillan St) & \\
        \textbf{office hours} & {by appointment\newline (\url{https://calendly.com/alexanderlerch})} & \\
    \end{tabular}
        
         
\section{General Information}        
    \subsection{Course Description}
        Introduction to the software-based analysis of digital music signals. This course covers the basic approaches for musical content analysis and teaches students to approach this class of problems and think algorithmically. Topics include pitch tracking, beat tracking, audio feature extraction, and genre classification. The classes focus is on the audio signal processing part of music information retrieval.
        
    \subsection{Prerequisites}
        The course will be open to any interested students in the Music Tech BS, MS, and PhD programs. Prior coursework or experience in (digital) signal processing and machine learning is expected. Programming experience and familiarity with (Matlab and) Python will be helpful. 

    \subsection{Learning Outcomes}        
        After successful completion of the class, the students will be able to 
        \begin{itemize}
            \item   summarize and explain baseline approaches to typical tasks in Music Information Retrieval,
            \item   describe and apply evaluation methods and metrics for audio content analysis systems,
            \item   implement audio analysis systems in Python, and
            \item   successfully complete a team project in conception, literature survey, proposal, implementation, evaluation, and presentation.
        \end{itemize}

\section{Grading}
    The following evaluative tools will be utilized in measuring progress towards obtaining the learning outcomes:
    \smallskip
    
    \begin{tabular}{lll}
        %\textbf{quizzes} & 20\%\\
        %\textbf{exercises} & 20\%\\
        \textbf{exercises \& assignments} & 45\%\\
        \textbf{exercises \& participation} & 10\%\\
        \textbf{project} & 45\% \\
        \quad presentation (proposal) & 5\%\\
        \quad presentation (midterm) & 5\%\\
        \quad presentation (final) & 5\%\\
        \quad paper & 10\%\\
        \quad algorithmic design and implementation & 20\%\\
    \end{tabular}
    
    \subsection{Description of Graded Components}
        If not explicitly mentioned otherwise, students in the undergraduate section of the class will work on the same components but with adjusted expectations for grading.
        \begin{itemize}
            \item   \textbf{exercises \& assignments: }\\
                Assignments will be posted according to the tentative schedule outlined in Sect.~\ref{sec:outline}. All assignments will contribute to the assignment grade with equal weight. If in-class exercises are part of the assignments, their grade will be part of the assignment grade. Assignments will be one or two questions shorter for the undergraduate section of the class.
            %\item   \textbf{exercises: }\\
                %Exercises will happen regularly and will be handed in after class.
            %\item   \textbf{quizzes: }\\
                %Quizzes might take place unannounced at any time during the semester and will assess your understanding of the content of video and in-class lectures. Each student will have one ``joker'', meaning that the worst result will be discarded and will not contribute to the overall grade.
            \item   \textbf{exercises \& participation:\\}
                Participation in class and in the forum with questions and answers as well as performance in in-class exercises not part of assignments. 
            \item   \textbf{project: }\\
                Each group (3 students) will work on a class project. The core of this project has to be a MIR algorithm developed by the team, but there is no additional restrictions: it might be a research project, an application for a specific task, a web service, etc. 
        \end{itemize}
    
    \subsection{Grading and Grading Policies}
        All graded components will be graded in points. The final grade for the course will be determined by dividing the total points earned by the number of points possible for each of the categories listed above. 

These numbers will be converted into a letter grade according to the following scale: 
    \smallskip
    
\begin{tabular}{ll}
    \textbf{A} & 100--90\%\\
    \textbf{B} & 89--80\%\\
    \textbf{C} & 79--70\%\\
    \textbf{D} & 69--60\%\\
    \textbf{F} & 59\% and below 
\end{tabular}
    \smallskip
    

Grades may be assigned per group or individually as announced (e.g., projects are in some cases per group, quizzes are usually per individual).
 

    \section{Course Materials}
        \subsection{Video Lectures}
            Videos of a previous iteration of the class are accessible at:\\            
            \url{https://www.audiocontentanalysis.org/class}
            
            The online video player should allow you to switch between slides and camera view.
            
        \subsection{Text Book}
            
           The text book for this class is available:
            \begin{quote}
            Alexander Lerch (2023), \textit{An Introduction to Audio Content Analysis: Music Information Retrieval Tasks and Applications}, 2nd Edition Jon Wiley \& Sons, Hoboken
            \end{quote}
            It can be accessed free of charge here (access to this site may be restricted from off-campus, use VPN):\\        
            \url{https://ieeexplore.ieee.org/book/9965970}
        
        %\subsection{Forum}
            %Questions and answers are encouraged online at:\\ 
            %\url{http://www.audiocontentanalysis.org/teaching/questions-answers}.
            
        \subsection{Additional (Optional) Reading}
            \begin{itemize}
                \item    Li, T., Ogihara, M. and Tzanetakis, G. (Eds.) (2012), \textit{Music Data Mining}. CRC Press 
                \item    Klapuri, A. and Davy, M. (Eds.) (2006), \textit{Signal Processing Methods for Music Transcription}. Springer 
                \item    Mueller, M. (2015), \textit{Fundamentals of Music Processing}. Springer
            \end{itemize}
        
        \subsection{Additional Resources}
            Additional resources are available on github. Additional resources include:
            \begin{itemize}
                \item   slides (pdf), \url{https://github.com/alexanderlerch/ACA-Slides}
                \item   links to Matlab code for plots in the slides, \url{https://github.com/alexanderlerch/ACA-Plots}
                \item   audio examples, \url{https://github.com/alexanderlerch/ACA-Slides/audio}
            \end{itemize}

        \subsection{Software}
                Assignments are due in the language announced (commonly python). The project work can be done in any programming language approved after discussion with the instructor. The most common choices are Python and Matlab. Matlab is accessible at \url{www.matlab.gatech.edu}.
    
    With respect to tools, \textbf{prepare to use github} (github.gatech.edu or github.com or some other version control system). I recommend using the github issues in connection with milestones to keep track of your project progress. Other recommended tools are
    \begin{itemize}
        \item   Zotero for bibliography management, and
        \item   \LaTeX\ for scientific typesetting.
    \end{itemize}
 

\section{Course Expectations \& Guidelines}
    \subsection{Course Schedule}\label{sec:outline}
        The class will be structured into the following parts: the lecture, the in-class exercises, the assignments, and the project work. The tentative schedule, subject to change, is shown in Table \ref{tab:schedule}.
        
        \bigskip
        
        \begin{table}
		\begin{tabular}{l|p{.3\textwidth}|p{.15\textwidth}|p{.175\textwidth}|p{.23\textwidth}}
            Week & Topics & Modules & potential in-class exercises & Assignment \& \textit{notes}\\
            \hline\hline
            1 & introduction & 0.0, 1.0, 2.0, 3.1 & project topic brainstorm& \\
            2 &  features & 3.2, 3.3, 3.4, 3.6 & spectrogram, correlation, features & ACF pitch tracking\\
            3 & inference& 3.7, 4& feature extraction & \textit{labor day}\\  %4
            4 & data and evaluation & 5, 6 & feature space visualization& feature extraction \& selection\\ %11.
            5 & project proposals &&& \\ % 18.
            6 & pitch tracking& 7.1, 7.2, 7.3, & HPS, AMDF& monophonic pitch trackers\\
            7 & pitch tracking and tuning frequency & 7.3, 7.4 & NMF, tuning freq &\\
            8 & key/chord detection, intensity & 7.5, 7.6, 8 &pitch chroma&   \textit{fall break}\\
            9 & mid-term project presentation &&& key/chord detection\\
            10 & onset,tempo detection& 9.1, 9.2, 9.3, 9.4, 9.5,9.6 & tempo detection & \\ %23.
            11 & structure, alignment& 9.7, 10 & DTW&\\
            12 &  genre classification & 12&& genre classification\\
            13 & audio fingerprinting, music similarity&11,13&& \\
            14 & mood classification &14& regression& \textit{thxgiving} \\ 
            15 & music performance assessment &16&& \\ % 
            16 & project work&&& \\
            %17 & --&& \\
            
		\end{tabular}
        \caption{planned class schedule}\label{tab:schedule}
        \end{table}

            Since all classes do not progress at the same rate, it may be necessary to modify the above schedule as circumstances dictate. For example, the number and frequency of assignments may be altered or the schedule of the classes may be changed. In either of these cases, adequate notification will be given and be discussed in class.
    

    
    \subsection{Academic Integrity}
        Georgia Tech aims to cultivate a community based on trust, academic integrity, and honor. Students are expected to act according to the highest ethical standards.  For information on Georgia Tech's Academic Honor Code, please visit  or .

        \begin{itemize}
            \item \url{http://www.catalog.gatech.edu/policies/honor-code/} or
            \item \url{http://www.catalog.gatech.edu/rules/18/}.
        \end{itemize}
        
Any student suspected of cheating or plagiarizing on a quiz, exam, or assignment will be reported to the Office of Student Integrity, who will investigate the incident and identify the appropriate penalty for violations.


    \subsection{Accommodations for Individuals with Disabilities}
                If you are a student with learning needs that require special accommodation, contact the Office of Disability Services (often referred to as ADAPTS) at (404)894-2563 or 
        \begin{itemize}
            \item \url{http://disabilityservices.gatech.edu}
        \end{itemize} 
        as soon as possible to make an appointment to discuss your special needs and to obtain an accommodations letter. Please also e-mail me as soon as possible in order to set up a time to discuss your learning needs.

    
    \subsection{Assignment Turn-In}
        All assignments and exercises have to be submitted to canvas unless announced otherwise. Each submission should contain source code as well as a document with plots and comments, structured by the tasks/questions as headers.\\
        The code and code documentation of the project work has to be turned in as a link to a online repository such as github.com or github.gatech.edu. 
    
    \subsection{Attendance and Participation}
                You are expected to attend the sessions unless you have a compelling reason not to do so. In any case active participation either synchronously or asynchronously is expected.
    
    %

        
    \subsection{Extensions, Late Assignments, Missed Exams}
                All assignments, papers, and other artifacts are due \textbf{ON THE DUE DATE}. The due date will be announced per assignment/task on t-square. A penalty of \textbf{TEN POINTS PER 24~HOURS} will be applied to all late assignments/tasks and late project papers. Documented illnesses and family emergencies are excepted. Quizzes and exams cannot be made up unless you have a valid, documented excuse.


    \subsection{Student Use of Mobile Devices in the Classroom}
                The use of mobile devices in the classroom is prohibited unless explicitly allowed by the instructor.

        
    \subsection{Student-Faculty Expectations}
                At Georgia Tech we believe that it is important to continually strive for an atmosphere of mutual respect, acknowledgment, and responsibility between faculty members and the student body. See 
        \begin{itemize}
            \item \url{http://www.catalog.gatech.edu/rules/22}
        \end{itemize} 
        for an articulation of some basic expectations --- that you can have of me, and that I have of you. In the end, simple respect for knowledge, hard work, and cordial interactions will help build the environment we seek. Therefore, I encourage you to remain committed to the ideals of Georgia Tech while in this class.

        
    \subsection{Diversity}
        The School of Music community of faculty, staff, and students aspires to create and nurture an environment that is supportive of all backgrounds where different views and ideas are respected and encouraged. In all our pursuits, we commit to justice, diversity, equity, and inclusion with regard to race, national origin, language, age, sexual orientation, gender, religion, and ability. Moreover, we will encourage intellectual inquiry and respectful exchange that cements our dedication to these principles.
        
    \subsection{Grievances and Concerns}
        Students should first discuss any concerns with the relevant faculty member; if it is not possible to come to resolution with the faculty member, students may then report the matter to the appropriate administrator (Chair or Associate Chair or Director of Studies) of the department of instruction or report it here: 
        \begin{itemize}
            \item \url{http://www.contact.gatech.edu/academicgrievance}
        \end{itemize} 
    The GT grievance policy can be found at 
        \begin{itemize}
            \item \url{https://provost.gatech.edu/reporting-units/conflict-resolution-ombuds/academic-grievance-policy}. 
        \end{itemize} 
    
    %Additionally, if you need formal assistance, please contact Associate Provost Jennifer Herazy (\url{mailto:herazy@gatech.edu}). For informal assistance or to speak to someone who can be a sounding board for you, please contact one of the Ombuds staff: Russ Callen (\url{mailto:russ.callen@ece.gatech.edu}) or Leigh Bottomley (\url{leigh.bottomley@gatech.edu}).


\end{document}

